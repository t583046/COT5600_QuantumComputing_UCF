\documentclass[12pt]{article}
\usepackage{amsmath, amssymb, amsthm,qcircuit}

\renewcommand{\>}{\rangle}
\newcommand{\<}{\langle}
\newcommand{\cL}{\mathcal{L}}
\newcommand{\cH}{\mathcal{H}}
\newcommand{\C}{\mathbb{C}}
\newcommand{\tr}{\mathrm{Tr}}

\setlength{\marginparwidth}{0pt} \setlength{\hoffset}{0cm}
\setlength{\oddsidemargin}{0pt} \setlength{\topmargin}{1cm}
\setlength{\headheight}{0pt} \setlength{\headsep}{0pt}
\addtolength{\textwidth}{3cm} \addtolength{\textheight}{5cm}

\begin{document}


\begin{center}
{\bf COT 5600 Quantum Computing} 

\medskip
{\bf Spring 2019}

\bigskip

{\bf Homework 1}
\end{center}

\newpage

%%%%%%%%%%%%%%%%%%%%%%%%%%%%%%%%%%%%%%%%% PROBLEM 1

\noindent {\bf Problem 1} (Eigenvalues of Pauli operators)

\medskip
\noindent
\newline\href{https://github.com/t583046}

\newpage

%%%%%%%%%%%%%%%%%%%%%%%%%%%%%%%%%%%%%%%%% PROBLEM 2

\noindent {\bf Problem 2} (Trace inner product)

\medskip
\noindent
Proof:
\newline
Assuming \C^{d\times d}$ is a real vector space

\newline
\newline
1) Using basic property of Trace: The matrix A and its Transpose have the same Trace. ie. 
\newline
\newline
\indent\indent\(tr(A)  = tr(A^T)\)
\newline
\newline

2)By definition of conjugate transpose, and assuming A has real entries, the conjugate transpose of A reduces to the transpose of A.
\newline
\newline
\indent\indent ie. from 2,
\newline
\newline\indent\(3) A^\dagger = tr(A^T) = tr(A)\). This is simply the trace of A.\newline

\indent4) This means that \(\< A | B\>_{\mathrm{Tr}} = \mathrm{Tr}(AB).\newline
\newline
 \indent\indent\(\< A | B\>_{\mathrm{Tr}} $  is also by definition the inner product trace. This Definition states that for real square matrices of the same size, \< A | B\>  = {\mathrm{Tr}}\((A B^T).\newline
 

\indent5)From 1, \(B^T = \) \(B\), thus \< A | B\>  = \mathrm{Tr}(AB).\newline


\newline\indent6) Therefore \(\mathrm{Tr}(AB) = \mathrm{Tr}(AB)\).\newline

\textbf{Q.E.D. \< A | B\>_{\mathrm{Tr}} = \mathrm{Tr}(A^\dagger B)\,}.
 
 
 


\newpage

%%%%%%%%%%%%%%%%%%%%%%%%%%%%%%%%%%%%%%%%% PROBLEM 3

\noindent {\bf Problem 3} (Unitary error basis)

\noindent
\newline\href{https://github.com/t583046}

\end{document}
